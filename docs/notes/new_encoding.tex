% ****** Start of file apssamp.tex ******
%
%   This file is part of the APS files in the REVTeX 4.1 distribution.
%   Version 4.1r of REVTeX, August 2010
%
%   Copyright (c) 2009, 2010 The American Physical Society.
%
%   See the REVTeX 4 README file for restrictions and more information.
%
% 
\documentclass[%
aps, %aps-style journal
prl, % physical review letters
preprint, % double spaced
12pt, % 12 point font
amsfonts, % ams fonts
amssymb, % ams symbols
amsmath, % ams math  features
endfloats,% put all figures/tables at  end, one per page
notitlepage, % start on the first page
raggedbottom, % allow variation at bottom
]{revtex4-1}

\usepackage{graphicx}% Include figure files
\usepackage{tikz}
\newcommand{\bra}[1]{|#1\rangle}
\newcommand{\ket}[1]{|#1\langle}
\newcommand\hlight[1]{\tikz[overlay, remember picture,baseline=-\the\dimexpr\fontdimen22\textfont2\relax]\node[rectangle,fill=blue!50,rounded corners,fill opacity = 0.2,draw,thick,text opacity =1] {$#1$};} 


\begin{document}

\preprint{}

\title{An efficient QAOA scheme for enconding undirected travelling salesman problem}
% Force line breaks with \\
\author{Yuning Zhang}

% \affiliation{SUSTech}

\date{\today}
\maketitle

\section{\label{sec:Intro}Introduction}
The qubits needed to encoding travelling salesman problem (TSP) into a QAOA Hamiltonian increase in $O(n^2)$ order under one-hot encoding, which is expensive for NISQ device with limited qubits. To make TSP solving with QAOA feasible, some techniques must be used to reduce the resource required for Hamiltonian encoding.

The first avaiable TSP encoding scheme is given by Hadfield et, al, where an enhanced QAOA ansatz is developed to tackle constrained optimization problems. Here we report a more resource-efficient scheme to encoding undirected TSP based on Hadfield's work, which could reduce the encoding qubits by 50\%.

For directed TSP, the solution is an arrangement for the order of all cities. The size of solution space for $n$-cities TSP is $A_n^n$. By fixing the start of the travel, the circle degeneracy is eliminated, with combination size reduced to $A_{n-1}^{n-1}$. If the TSP is undirected, which means $d_{i,j}=d_{j,i}$, the cost of a specific tour is equal to its reverse. Remove this degeneracy and the solution space size can be reduced to $A_{(n-1)/2}^{n-1}$.

Eg.
$0\rightarrow3\rightarrow1\rightarrow2\rightarrow4$ is a solution for a 5 cities TSP,
in undirected conditions, it equals to 
$0\rightarrow4\rightarrow2\rightarrow1\rightarrow3$


Target Hamiltonian:
$$
H_{cost}=\sum_{i,j} d_{i,j} \hat p_{i,j}=\frac{1}{2}\sum_{i,j} d_{i,j} (I-Z_{i,j})
$$
Instead of encoding vertices as Hadfield did, here we encoding the edges of a TSP into the cost Hamiltonian. Where $\hat p_{i,j}=1/2(I-Z_{i,j})$ indicates wether the edge $(i,j)$ is chosen. Since the setup of a TSP is completely represented by its distance matrix, containing $n(n-1)$ non zero elements, we just need the same number of qubits to encoding each edge of the graph. But it should be noticed that the encoding scheme reserves huge redundancy. The combination space for edge coding is $C_{n(n-1)}^n$, which is much larger than $A_{n-1}^{n-1}$. To form a circle requires in TSP, we need to pick $n$ edges, and add constraints that $\forall\, i\in S, deg(i)=2$, ensuring that each city is passed. 

We can use the same operator ansatz technique to preserve these constraints.

Ansatz:
$$
|\psi_0\rangle=|\bar 0\rangle \otimes_{i=1}^{n-1} |1_{i,i+1}\rangle\otimes\bra{1_{n,0}}
$$
The ansatz is a valid solution for TSP.

Mixer Hamiltonian:
$$
H_{mixer}=\sum_{\{i,j\},\{u,v\}} S^-_{u,i}S^-_{i,j}S^-_{j,v}S^+_{u,j}S^+_{j,i}S^+_{i,v}+S^+_{u,i}S^+_{i,j}S^+_{j,v}S^-_{u,j}S^-_{j,i}S^-_{i,v}
$$
The mixer Hamiltonian is the \textit{adjancent swap} operator of solution order, where $S^+=|1\rangle\langle 0|$ and $S^-=|0\rangle\langle 1|$.
The operator will swap the order of two adjancent cities, eg.
$|u,i,j,v\rangle \leftrightarrow |u,j,i,v\rangle$, with corresponding cost transfer $d_{u,i}+d_{i,j}+d_{j,v} \leftrightarrow d_{u,j}+d_{j,i}+d_{i,v}$.

For example the $5\times 5$ matrix below.
$$
\begin{bmatrix}
    0&\hlight{d_{0,1}}&d_{0,2}&d_{0,3}&d_{0,4}&\\
    d_{1,0}&0&\hlight{d_{1,2}}&d_{1,3}&d_{1,4}&\\
    d_{2,0}&d_{2,1}&0&\hlight{d_{2,3}}&0d_{2,4}&\\
    d_{3,0}&d_{3,1}&d_{3,2}&0&\hlight{d_{3,4}}&\\
    \hlight{d_{4,0}}&d_{4,1}&d_{4,2}&d_{4,3}&0&\\
\end{bmatrix}
$$
The ansatz is $\bra{0\rightarrow 1 \rightarrow 2\rightarrow 3\rightarrow 4}$, with cost $\sum_{i=0}^4 d_{i,i+1}$ where $d_{n,n+1}=d_{n,0}$.

This encoding scheme is equivalent to the original operator ansatz, where at least $(n-1)^2$ qubits is needed. For the edge encoding, we need to map every non-zero matrix elements to a qubit thus $n(n-1)$ qubits are required. But if we take the special situation where the TSP is undirected, the new encoding scheme will only requires $n(n-1)/2$ qubits since we can map index $\{i,j\}$ and $\{j,i\}$ to one same qubit which represents an edge.


\section{Reference}
[1] S. Hadfield, Z. Wang, B. O’Gorman, E. G. Rieffel, D. Venturelli, and R. Biswas, Algorithms 12, 34 (2019).

\end{document}
%
% ****** End of file apssamp.tex ******