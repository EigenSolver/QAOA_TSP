\documentclass[a4paper, amsfonts, amssymb, amsmath, reprint, showkeys, nofootinbib, twoside]{revtex4-1}
\usepackage[english]{babel}
\usepackage[utf8]{inputenc}
\usepackage[colorinlistoftodos, color=green!40, prependcaption]{todonotes}
\input{preamble}
\usepackage[pdftex, pdftitle={Article}, pdfauthor={Author}]{hyperref} % For hyperlinks in the PDF
%\setlength{\marginparwidth}{2.5cm}
\bibliographystyle{apsrev4-1}
\begin{document}
\title{My Title}

\author{Author}
    \email[Correspondence email address: ]{email@institution.com}% Your name
    \affiliation{University of São Paulo, Institute of Physics, São Paulo, SP, Brazil}

\date{\today} % Leave empty to omit a date

\begin{abstract}
\lipsum[1]
\end{abstract}

\keywords{first keyword, second keyword, third keyword}


\maketitle

\input{sections/section01.tex}  %I believe leaving the sections in separate files is more organized, change it if you desire 
\input{sections/section02.tex}
\input{sections/section03.tex}
\input{sections/acknowledgements.tex}

\begin{thebibliography}{4}
\bibitem{Griffiths}
D. J. Griffiths,
\textit{Introduction to Electrodynamics}
(Cambridge University Press, Cambridge, 2017).

\bibitem{Fleming}
A. Bobrinha,
Revista Brasileira de Lorem Ipsum \textbf{23},
179 (2002).

\bibitem{Feynman}
R. P. Feynman, R. B. Leighton and M. Sands,
\textit{Lições de Física de Feynman}
(Editora Bookman, Porto Alegre, 2008).

\bibitem{Jackson-CE}
J. D. Jackson,
\textit{Classical Electrodynamics}
(John Wiley \& Sons, Danvers, 1999).
\end{thebibliography}

\appendix*
\input{sections/appendix1.tex}

\end{document}